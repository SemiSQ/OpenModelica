\documentclass[a4paper]{report}
\usepackage[english]{babel}
\usepackage[latin1]{inputenc}

\author{David K�gedal}
\title{Blahonga Blahonga}

\begin{document}

\titlepage
\maketitle{}
\endtitlepage


\begin{abstract}
\label{abs}

Formal descriptions of the syntax of programming languages has long
been accepted as a natural way of describing the syntacical form of
the language. Today everobody expects a BNF-like grammar for a new
language. But that is not the case when it comes to describing the
semantics of languages. Formalisms for the specifications of semantics
hase been avaiable for many years, but are not often used by others
than researchers.

One advantage of using a formal description of the semantics is of
course that the specification becomes more strict and unambiguous. But
there is also the possibility of using the semantic specification to
generate a language translator, or compiler, in an automatic way.

This paper describes a formal semantics for Modelica, an
equation-based language used for modeling. The notation used enables
it to be passed to a compiler generator that generates a Modelica
translator.

\end{abstract}

\tableofcontents

\chapter{Background}
\label{cha:bg}



\end{document}
