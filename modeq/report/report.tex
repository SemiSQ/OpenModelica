\documentclass[a4paper]{report}
\usepackage[english]{babel}
\usepackage[latin1]{inputenc}

\author{David K�gedal}
\title{Blahonga Blahonga}

\newcommand{\code}[1]{\texttt{#1}}
\newcommand{\firstref}[1]{\textit{#1}}

\newtheorem{Def}{Definition}[chapter]

\begin{document}

\titlepage
\maketitle{}
\endtitlepage


\begin{abstract}
\label{abs}

Formal descriptions of the syntax of programming languages has long
been accepted as a natural way of describing the syntacical form of
the language. Today everobody expects a BNF-like grammar for a new
language. But that is not the case when it comes to describing the
semantics of languages. Formalisms for the specifications of semantics
hase been avaiable for many years, but are not often used by others
than researchers.

One advantage of using a formal description of the semantics is of
course that the specification becomes more strict and unambiguous. But
there is also the possibility of using the semantic specification to
generate a language translator, or compiler, in an automatic way.

This paper describes a formal semantics for Modelica, an
equation-based language used for modeling. The notation used enables
it to be passed to a compiler generator that generates a Modelica
translator.

\end{abstract}

\tableofcontents

\chapter{Background}
\label{cha:bg}


\chapter{Semantics}
\label{cha:semantics}



\section{Types}
\label{sec:types}



\section{Connections}
\label{sec:connections}

Connections between objects are introduced by the \code{connect}
construct in the equation part of a class declaration.  The
\code{connect} construct takes two references to connectors in the
same class or in one of its components and connects them according to
the rules below.  Each connector reference has either the syntactic
form \code{c}, where \code{c} is a connector instance in the class
containing the \code{connect} construct, or \code{m.c}, where \code{m}
is the name of a component or an array of components (??) of the class
containing the \code{connect} construct, and \code{c} is the name of a
connector variable in the component \code{m}.


\subsection{Connection sets}
\label{sec:connectsets}

To create equations from the \code{connect} constructs, the
connections are collected in \firstref{connection sets}.

\begin{Def}[Connection sets]
  A connection set is a set $C$ of variables connected by means of
  \code{connect} constructs.
\end{Def}

When a \code{connect(a,b)} construct is encountered, the component
names \code{a} and \code{b} are checked to be of the same type.

\subsection{Equations}
\label{sec:coneq}

In the case of variables declared with the \code{flow} type modifier,
the equation generated is a sum-to-zero equation. \[\sum_{v \in C} d_v
v = 0\]

\end{document}
